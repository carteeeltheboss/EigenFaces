\documentclass{beamer}
\usepackage{listings}
\usepackage{xcolor}
\usepackage{amsmath}

% Configure listings for Python code
\definecolor{mygray}{rgb}{0.5,0.5,0.5}
\definecolor{mygreen}{rgb}{0,0.6,0}
\definecolor{mymauve}{rgb}{0.58,0,0.82}
\lstset{%
  language=Python,
  basicstyle=\tiny\ttfamily,
  numbers=left,
  numberstyle=\tiny\color{mygray},
  stepnumber=1,
  numbersep=5pt,
  backgroundcolor=\color{white},
  showspaces=false,
  showstringspaces=false,
  showtabs=false,
  frame=single,
  tabsize=2,
  captionpos=b,
  breaklines=true,
  breakatwhitespace=false,
  keywordstyle=\color{blue},
  commentstyle=\color{mygreen},
  stringstyle=\color{mymauve},
}

\title{Facial Recognition with PCA (Eigenfaces) --- From Scratch}
\author{Your Name}
\institute{Your Institution}
\date{\today}

\begin{document}

\begin{frame}
  \titlepage
\end{frame}

\begin{frame}
  \frametitle{Outline}
  \tableofcontents
\end{frame}

%%%%%%%%%%%%%%%%%%%%%%%%%%%%%%%%%%%%%%%%%%%%%%%%%%%%%%%%%%%%%%%%%%%%%%
\section{Introduction}
%%%%%%%%%%%%%%%%%%%%%%%%%%%%%%%%%%%%%%%%%%%%%%%%%%%%%%%%%%%%%%%%%%%%%%

\begin{frame}
  \frametitle{Introduction}
  \begin{itemize}
    \item This presentation explains a facial recognition system using PCA (Eigenfaces) implemented entirely from scratch.
    \item No external libraries like NumPy or scikit-learn are used; only built-in Python modules, \texttt{math}, and \texttt{matplotlib} for visualization.
    \item The code demonstrates:
      \begin{itemize}
        \item Basic mathematical operations and matrix arithmetic.
        \item Image loading and preprocessing.
        \item PCA calculation via the covariance trick and power iteration.
        \item Projection of images and matching using Euclidean distance.
      \end{itemize}
  \end{itemize}
\end{frame}

%%%%%%%%%%%%%%%%%%%%%%%%%%%%%%%%%%%%%%%%%%%%%%%%%%%%%%%%%%%%%%%%%%%%%%
\section{Helper Math Functions and Matrix Operations}
%%%%%%%%%%%%%%%%%%%%%%%%%%%%%%%%%%%%%%%%%%%%%%%%%%%%%%%%%%%%%%%%%%%%%%

\begin{frame}[fragile]
  \frametitle{Helper Math Functions}
  \begin{itemize}
    \item \texttt{dot(v, w)}: Computes the dot product of two vectors.
    \item \texttt{vector\_add(v, w)}: Adds two vectors elementwise.
    \item \texttt{vector\_sub(v, w)}: Subtracts two vectors elementwise.
    \item \texttt{scalar\_mult(v, s)}: Multiplies each element of a vector by a scalar.
    \item \texttt{norm(v)}: Computes the Euclidean norm.
  \end{itemize}
  \vspace{5mm}
  These functions allow us to perform linear algebra calculations using only Python lists.
\end{frame}

\begin{frame}[fragile]
  \frametitle{Matrix Multiplication}
  \begin{lstlisting}
def mat_mult(A, B):
    """Multiply two matrices A and B."""
    m = len(A)
    n = len(A[0])
    p = len(B[0])
    result = [[0.0 for _ in range(p)] for _ in range(m)]
    for i in range(m):
        for j in range(p):
            for k in range(n):
                result[i][j] += A[i][k] * B[k][j]
    return result
  \end{lstlisting}
  This function implements matrix multiplication using triple for‑loops.
\end{frame}

%%%%%%%%%%%%%%%%%%%%%%%%%%%%%%%%%%%%%%%%%%%%%%%%%%%%%%%%%%%%%%%%%%%%%%
\section{Image Loading and Processing}
%%%%%%%%%%%%%%%%%%%%%%%%%%%%%%%%%%%%%%%%%%%%%%%%%%%%%%%%%%%%%%%%%%%%%%

\begin{frame}[fragile]
  \frametitle{Image Loading \& Processing}
  \begin{itemize}
    \item \texttt{load\_image(filepath)}: Uses \texttt{matplotlib.image.imread} to load an image and converts it to grayscale if needed.
    \item \texttt{flatten\_image(img)}: Flattens a 2D image into a 1D list.
    \item \texttt{load\_dataset(folder)}: Loads all image files from a folder and extracts labels from filenames.
  \end{itemize}
  \vspace{3mm}
  The images must be in folders (e.g., ``train'' and ``test'') with filenames starting with a label (e.g., \texttt{0\_face1.png}).
\end{frame}

%%%%%%%%%%%%%%%%%%%%%%%%%%%%%%%%%%%%%%%%%%%%%%%%%%%%%%%%%%%%%%%%%%%%%%
\section{PCA Computation}
%%%%%%%%%%%%%%%%%%%%%%%%%%%%%%%%%%%%%%%%%%%%%%%%%%%%%%%%%%%%%%%%%%%%%%

\begin{frame}[fragile]
  \frametitle{Computing the Mean Face}
  \begin{lstlisting}
def compute_mean_face(images):
    n = len(images)
    m = len(images[0])
    mean_face = [0.0] * m
    for face in images:
        for i in range(m):
            mean_face[i] += face[i]
    mean_face = [x / n for x in mean_face]
    return mean_face
  \end{lstlisting}
  This function averages all training images to produce the \emph{mean face}.
\end{frame}

\begin{frame}[fragile]
  \frametitle{Centering the Images}
  \begin{lstlisting}
def center_images(images, mean_face):
    centered = []
    for face in images:
        centered_face = [face[i] - mean_face[i] for i in range(len(face))]
        centered.append(centered_face)
    return centered
  \end{lstlisting}
  Each image is centered by subtracting the mean face.
\end{frame}

\begin{frame}[fragile]
  \frametitle{Covariance Matrix (Using the Trick)}
  \begin{lstlisting}
def compute_covariance_matrix(X):
    n = len(X)
    L = [[0.0 for _ in range(n)] for _ in range(n)]
    for i in range(n):
        for j in range(n):
            L[i][j] = dot(X[i], X[j])
    return L
  \end{lstlisting}
  Instead of working with a huge covariance matrix in pixel space, we compute the smaller matrix \(L = XX^T\).
\end{frame}

%%%%%%%%%%%%%%%%%%%%%%%%%%%%%%%%%%%%%%%%%%%%%%%%%%%%%%%%%%%%%%%%%%%%%%
\section{Eigen Decomposition}
%%%%%%%%%%%%%%%%%%%%%%%%%%%%%%%%%%%%%%%%%%%%%%%%%%%%%%%%%%%%%%%%%%%%%%

\begin{frame}[fragile]
  \frametitle{Power Iteration}
  \begin{lstlisting}
def power_iteration(A, num_iter=100, tol=1e-6):
    n = len(A)
    v = [1.0 for _ in range(n)]
    v_norm = norm(v)
    v = [x / v_norm for x in v]
    for _ in range(num_iter):
        Av = [sum(A[i][j] * v[j] for j in range(n)) for i in range(n)]
        Av_norm = norm(Av)
        new_v = [x / Av_norm for x in Av]
        diff = sum(abs(new_v[i] - v[i]) for i in range(n))
        v = new_v
        if diff < tol:
            break
    eigenvalue = dot(v, [sum(A[i][j] * v[j] for j in range(n)) for i in range(n)])
    return eigenvalue, v
  \end{lstlisting}
  This function estimates the dominant eigenvalue/eigenvector pair via power iteration.
\end{frame}

\begin{frame}[fragile]
  \frametitle{Deflation}
  \begin{lstlisting}
def deflate(A, eigenvalue, eigenvector):
    n = len(A)
    for i in range(n):
        for j in range(n):
            A[i][j] -= eigenvalue * eigenvector[i] * eigenvector[j]
    return A
  \end{lstlisting}
  Once an eigenpair is found, deflation removes its influence from the matrix.
\end{frame}

\begin{frame}[fragile]
  \frametitle{Extracting Top \(k\) Eigenpairs}
  \begin{lstlisting}
def top_k_eigen(L, k):
    eigenvalues = []
    eigenvectors = []
    A = [row[:] for row in L]
    for _ in range(k):
        eigenvalue, evector = power_iteration(A)
        eigenvalues.append(eigenvalue)
        eigenvectors.append(evector)
        A = deflate(A, eigenvalue, evector)
    return eigenvalues, eigenvectors
  \end{lstlisting}
  Iteratively finds the top \(k\) eigenvalues and eigenvectors of the covariance trick matrix \(L\).
\end{frame}

\begin{frame}[fragile]
  \frametitle{Computing Eigenfaces}
  \begin{lstlisting}
def compute_eigenfaces(X_centered, eigenvectors):
    n = len(X_centered)
    m = len(X_centered[0])
    eigenfaces = []
    for v in eigenvectors:
        face = []
        for j in range(m):
            val = sum(X_centered[i][j] * v[i] for i in range(n))
            face.append(val)
        f_norm = norm(face)
        if f_norm != 0:
            face = [x / f_norm for x in face]
        eigenfaces.append(face)
    return eigenfaces
  \end{lstlisting}
  The eigenvectors of \(L\) are back-projected into the original pixel space to form the eigenfaces.
\end{frame}

%%%%%%%%%%%%%%%%%%%%%%%%%%%%%%%%%%%%%%%%%%%%%%%%%%%%%%%%%%%%%%%%%%%%%%
\section{Projection and Recognition}
%%%%%%%%%%%%%%%%%%%%%%%%%%%%%%%%%%%%%%%%%%%%%%%%%%%%%%%%%%%%%%%%%%%%%%

\begin{frame}[fragile]
  \frametitle{Projecting Images}
  \begin{lstlisting}
def project_image(image, eigenfaces):
    return [dot(image, ef) for ef in eigenfaces]
  \end{lstlisting}
  Projects a centered image into the reduced eigenface space by taking dot products with each eigenface.
\end{frame}

\begin{frame}[fragile]
  \frametitle{Matching Faces}
  \begin{lstlisting}
def euclidean_distance(v1, v2):
    return math.sqrt(sum((a - b) ** 2 for a, b in zip(v1, v2)))

def match_face(test_proj, train_projs):
    best_index = 0
    best_distance = euclidean_distance(test_proj, train_projs[0])
    for i in range(1, len(train_projs)):
        d = euclidean_distance(test_proj, train_projs[i])
        if d < best_distance:
            best_distance = d
            best_index = i
    return best_index
  \end{lstlisting}
  These functions compute the Euclidean distance between the test image and all training images to find the best match.
\end{frame}

%%%%%%%%%%%%%%%%%%%%%%%%%%%%%%%%%%%%%%%%%%%%%%%%%%%%%%%%%%%%%%%%%%%%%%
\section{Main Routine Overview}
%%%%%%%%%%%%%%%%%%%%%%%%%%%%%%%%%%%%%%%%%%%%%%%%%%%%%%%%%%%%%%%%%%%%%%

\begin{frame}[fragile]
  \frametitle{Main Routine: Data Preparation}
  \begin{lstlisting}
def main():
    # Specify folders with images (filenames begin with label)
    train_folder = "train"
    test_folder = "test"
    
    # Load the datasets
    train_images, train_labels = load_dataset(train_folder)
    test_images, test_labels = load_dataset(test_folder)
    
    # Compute and subtract the mean face
    mean_face = compute_mean_face(train_images)
    train_centered = center_images(train_images, mean_face)
    test_centered = center_images(test_images, mean_face)
  \end{lstlisting}
  Loads images, extracts labels, and centers the data by subtracting the mean face.
\end{frame}

\begin{frame}[fragile]
  \frametitle{Main Routine: PCA and Eigenface Computation}
  \begin{lstlisting}
    # Compute covariance matrix via the "trick"
    L = compute_covariance_matrix(train_centered)
    
    # Extract top eigenpairs (e.g., using 5 components for demo)
    num_components = min(5, len(train_centered))
    eigen_vals, eigen_vecs = top_k_eigen(L, num_components)
    
    # Form the eigenfaces
    eigenfaces = compute_eigenfaces(train_centered, eigen_vecs)
  \end{lstlisting}
  This section performs the PCA: forming the covariance matrix, computing eigenpairs, and finally generating eigenfaces.
\end{frame}

\begin{frame}[fragile]
  \frametitle{Main Routine: Projection, Recognition, \& Evaluation}
  \begin{lstlisting}
    # Project images into eigenface space
    train_projections = [project_image(face, eigenfaces) for face in train_centered]
    test_projections  = [project_image(face, eigenfaces) for face in test_centered]
    
    # Identify best match for each test image
    predicted = []
    for test_proj in test_projections:
        idx = match_face(test_proj, train_projections)
        predicted.append(train_labels[idx])
    
    # Compute recognition accuracy
    correct = sum(1 for a, b in zip(predicted, test_labels) if a == b)
    accuracy = correct / len(test_labels)
    print("Facial recognition accuracy: {:.2f}%".format(accuracy * 100))
  \end{lstlisting}
  Projects each test image, compares them with training projections, and computes the accuracy.
\end{frame}

\begin{frame}[fragile]
  \frametitle{Main Routine: Visualization}
  \begin{lstlisting}
    # Determine image dimensions (assumed square)
    img_dim = int(math.sqrt(len(train_images[0])))
    num_display = min(3, len(test_images))
    fig, axes = plt.subplots(2, num_display, figsize=(10, 5))
    
    for i in range(num_display):
        # Display the test image
        test_img = test_images[i]
        axes[0, i].imshow([test_img[j*img_dim:(j+1)*img_dim] for j in range(img_dim)], cmap='gray')
        axes[0, i].set_title("Test label: " + str(test_labels[i]))
        axes[0, i].axis('off')
        
        # Find and display the best matching training image
        match_idx = match_face(test_projections[i], train_projections)
        train_img = train_images[match_idx]
        axes[1, i].imshow([train_img[j*img_dim:(j+1)*img_dim] for j in range(img_dim)], cmap='gray')
        axes[1, i].set_title("Matched: " + str(train_labels[match_idx]))
        axes[1, i].axis('off')
        
    plt.suptitle("Facial Recognition Results")
    plt.show()

if __name__ == '__main__':
    main()
  \end{lstlisting}
  The final portion visualizes a few test images and their best-matched training images.
\end{frame}

%%%%%%%%%%%%%%%%%%%%%%%%%%%%%%%%%%%%%%%%%%%%%%%%%%%%%%%%%%%%%%%%%%%%%%
\section{Conclusion}
%%%%%%%%%%%%%%%%%%%%%%%%%%%%%%%%%%%%%%%%%%%%%%%%%%%%%%%%%%%%%%%%%%%%%%

\begin{frame}
  \frametitle{Conclusion}
  \begin{itemize}
    \item We implemented a facial recognition system using PCA (Eigenfaces) entirely from scratch.
    \item The approach includes custom linear algebra functions, image preprocessing, PCA via power iteration, and recognition by Euclidean distance.
    \item While educational, this code is not optimized for performance; optimized libraries should be used in production.
  \end{itemize}
\end{frame}

\begin{frame}
  \frametitle{Questions}
  \centering
  \Large Thank you! \\
  \vspace{10mm}
  Questions?
\end{frame}

\end{document}
